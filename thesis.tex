
% Cal Poly Thesis
% 
% based on UC Thesis format
%
% modified by Mark Barry 2/07.
%




\documentclass[12pt]{ucthesis}

\newif\ifpdf
\ifx\pdfoutput\undefined
    \pdffalse % we are not running PDFLaTeX
\else
\pdfoutput=1 % we are running PDFLaTeX
\pdftrue \fi

\usepackage{url}
\ifpdf

    \usepackage[pdftex]{graphicx}
    % Update title and author below...
    \usepackage[pdftex,plainpages=false,breaklinks=true,colorlinks=true,urlcolor=blue,citecolor=blue,%
                                       linkcolor=blue,bookmarks=true,bookmarksopen=true,%
                                       bookmarksopenlevel=3,pdfstartview=FitV,
                                       pdfauthor={!!Author goes here!!},
                                       pdftitle={!!Title goes here!!},
                                       pdfkeywords={thesis, masters, cal poly}
                                       ]{hyperref}
    %Options with pdfstartview are FitV, FitB and FitH
    \pdfcompresslevel=1

\else
    \usepackage{graphicx}
\fi

\usepackage{amssymb}
\usepackage{amsmath}
\usepackage[letterpaper]{geometry}
\usepackage[overload]{textcase}



\bibliographystyle{abbrv}

\setlength{\parindent}{0.25in} \setlength{\parskip}{6pt}

\geometry{verbose,nohead,tmargin=1.25in,bmargin=1in,lmargin=1.5in,rmargin=1.3in}

\setcounter{tocdepth}{2}


% Different font in captions (single-spaced, bold) ------------
\newcommand{\captionfonts}{\small\bf\ssp}

\makeatletter  % Allow the use of @ in command names
\long\def\@makecaption#1#2{%
  \vskip\abovecaptionskip
  \sbox\@tempboxa{{\captionfonts #1: #2}}%
  \ifdim \wd\@tempboxa >\hsize
    {\captionfonts #1: #2\par}
  \else
    \hbox to\hsize{\hfil\box\@tempboxa\hfil}%
  \fi
  \vskip\belowcaptionskip}
\makeatother   % Cancel the effect of \makeatletter
% ---------------------------------------




\begin{document}

% Declarations for Front Matter

% Update fields below!
\title{Real Time Visibility Culling With Hardware Occlusion Queries and Uniform Grids}
\author{Ilya Seletsky}
\degreemonth{June} \degreeyear{2013} \degree{Master of Science}
\defensemonth{June} \defenseyear{2013}
\numberofmembers{3} \chair{Zo\"{e} Wood, Ph.D.} \othermemberA{Foaad Khosmood, Ph.D.} \othermemberB{Aaron Keen, Ph.D.} \field{Computer Science} \campus{San Luis Obispo}
\copyrightyears{seven}	%What does this mean?

\maketitle

\begin{frontmatter}

% Custom made for Cal Poly (by Mark Barry, modified by Andrew Tsui).
\copyrightpage

% Custom made for Cal Poly (by Andrew Tsui).
\committeemembershippage

\begin{abstract}

Culling out non-visible portions of 3D scenes is important for rendering large complex worlds at an interactive frame rate.  Past 3D engines used static prebaked visibility data which was generated using complex algorithms.  Hardware Occlusion Queries are a modern feature that allows engines to determine if objects are invisible on the fly.  This allows for fully dynamic destructible and editable environments as opposed to static prebaked environments of the past.  This paper presents an algorithm that uses Hardware Occlusion Queries to cull fully dynamic scenes in real time.  This algorithm is relatively simple in comparison to other real time occlusion culling techniques, making it possible for the average developer to render large detailed scenes.  It also requires very little work from the artists who design the scenes since no portals, occluders, or other special objects need to be used.

\end{abstract}

%\begin{acknowledgements}

%   Thank you...

%\end{acknowledgements}


\tableofcontents


\listoftables

\listoffigures

\end{frontmatter}

\pagestyle{plain}




\renewcommand{\baselinestretch}{1.66}


% ------------- Main chapters here --------------------





\chapter{Introduction}
\label{intro}

\section{Real Time Graphics}
\label{real-time-graphics}

Real time computer graphics are needed for CAD applications, simulations, video games, and even for just displaying the user interface of an operating system.  Computer graphics can be non-real time, like when doing CGI for a movie.  Each frame can take hours to render, and the result is a very photorealistic image that looks like it?s part of the movie itself.  Millions of frames are generated only for each still frame to be visible for a fraction of a second until the next frame to give the illusion of motion.  Real time graphics also display a frame for only a fraction of a second but are rendered on the fly right as the user is controlling the application.  This means the computer can?t sit there for hours rendering a frame, it has 33.33 milliseconds if trying to render at 30 frames per second.  Movies are typically shown at 25 FPS (frames per second), but 30 FPS is about the threshold until us humans start to notice slightly unresponsive input and jerky movement when directly controlling a real time graphics application.  Most computer monitors update at 60 Hz so real time graphics typically strive for 60 FPS giving an even smoother experience, and this means the computer now only has 16.66 milliseconds to render a frame.

\chapter{Previous Work}
\label{previous-work}

LaTeX is a document markup language and document preparation system for the TeX typesetting program.

It is widely used by mathematicians, scientists, philosophers, engineers, and scholars in academia and the commercial world, and by others as a primary or intermediate format (e.g. translating DocBook and other XML-based formats to PDF) because of the quality of typesetting achievable by TeX. It offers programmable desktop publishing features and extensive facilities for automating most aspects of typesetting and desktop publishing, including numbering and cross-referencing, tables and figures, page layout and bibliographies.

LaTeX is intended to provide a high-level language to access the power of TeX. LaTeX essentially comprises a collection of TeX macros, and a program to process LaTeX documents. Since TeX's formatting commands are very low-level, it is usually much simpler for end-users to use LaTeX.

LaTeX was originally written in 1984 by Leslie Lamport at SRI International and has become the dominant method for using TeX�few people write in plain TeX anymore.

LaTeX is based on the idea that authors should be able to focus on the meaning of what they are writing, without being distracted by the visual presentation of the information. In preparing a LaTeX document, the author specifies the logical structure using familiar concepts such as chapter, section, table, figure, etc., and lets the LaTeX system worry about the presentation of these structures. It therefore encourages the separation of layout from content, while still allowing manual typesetting adjustments where needed. This is similar to the mechanism by which many word processors allow styles to be defined globally for an entire document, or the CSS mechanism used by HTML.

\section{This is a new section}
\label{a-new-section}

LaTeX is a document markup language and document preparation system for the TeX typesetting program.

It is widely used by mathematicians, scientists, philosophers, engineers, and scholars in academia and the commercial world, and by others as a primary or intermediate format (e.g. translating DocBook and other XML-based formats to PDF) because of the quality of typesetting achievable by TeX. It offers programmable desktop publishing features and extensive facilities for automating most aspects of typesetting and desktop publishing, including numbering and cross-referencing, tables and figures, page layout and bibliographies.

LaTeX is intended to provide a high-level language to access the power of TeX. LaTeX essentially comprises a collection of TeX macros, and a program to process LaTeX documents. Since TeX's formatting commands are very low-level, it is usually much simpler for end-users to use LaTeX.

\begin{figure}
\begin{center}
\[
\left[\begin{array}{ccc}
u_{x} & u_{y} & u_{z}\\
v_{x} & v_{y} & v_{z}\\
w_{x} & w_{y} & w_{z}\end{array}\right]\left[\begin{array}{c}
p_{x}\\
p_{y}\\
p_{z}\end{array}\right]=\left[\begin{array}{c}
p_{x}^{\prime}\\
p_{y}^{\prime}\\
p_{z}^{\prime}\end{array}\right]\]
\captionfonts
\caption[A matrix equation]{This is a sample matrix equation.}
\label{eqn:example}
\end{center}
\end{figure}


LaTeX was originally written in 1984 by Leslie Lamport at SRI International and has become the dominant method for using TeX�few people write in plain TeX anymore.

LaTeX is based on the idea that authors should be able to focus on the meaning of what they are writing, without being distracted by the visual presentation of the information. In preparing a LaTeX document, the author specifies the logical structure using familiar concepts such as chapter, section, table, figure, etc., and lets the LaTeX system worry about the presentation of these structures. It therefore encourages the separation of layout from content, while still allowing manual typesetting adjustments where needed. This is similar to the mechanism by which many word processors allow styles to be defined globally for an entire document, or the CSS mechanism used by HTML.


\begin{enumerate}
\item Here is a list item.
\begin{enumerate}
\item Here is a sub list item.
\begin{enumerate}
\item Here is a sub sub list item.

\end{enumerate}
\end{enumerate}
\end{enumerate}



\chapter{Results}
\label{results}

Here is the results section.

LaTeX is a document markup language and document preparation system for the TeX typesetting program.

It is widely used by mathematicians, scientists, philosophers, engineers, and scholars in academia and the commercial world, and by others as a primary or intermediate format (e.g. translating DocBook and other XML-based formats to PDF) because of the quality of typesetting achievable by TeX. It offers programmable desktop publishing features and extensive facilities for automating most aspects of typesetting and desktop publishing, including numbering and cross-referencing, tables and figures, page layout and bibliographies.

LaTeX is intended to provide a high-level language to access the power of TeX. LaTeX essentially comprises a collection of TeX macros, and a program to process LaTeX documents. Since TeX's formatting commands are very low-level, it is usually much simpler for end-users to use LaTeX.

LaTeX was originally written in 1984 by Leslie Lamport at SRI International and has become the dominant method for using TeX�few people write in plain TeX anymore.

LaTeX is based on the idea that authors should be able to focus on the meaning of what they are writing, without being distracted by the visual presentation of the information. In preparing a LaTeX document, the author specifies the logical structure using familiar concepts such as chapter, section, table, figure, etc., and lets the LaTeX system worry about the presentation of these structures. It therefore encourages the separation of layout from content, while still allowing manual typesetting adjustments where needed. This is similar to the mechanism by which many word processors allow styles to be defined globally for an entire document, or the CSS mechanism used by HTML.

LaTeX can be arbitrarily extended by using the underlying macro language to develop custom formats. Such macros are often collected into packages which are available to address special formatting issues such as complicated mathematical content or graphics. In addition, there are numerous commercial implementations of the entire TeX system, including LaTeX, to which vendors may add extra features like additional typefaces and telephone support. LyX is a free visual document processor that uses LaTeX for a back-end. TeXmacs is a free, WYSIWYG editor with similar functionalities as LaTeX, but a different typesetting engine.

A number of popular commercial desktop publishing systems use modified versions of the original TeX typesetting engine. The recent rise in popularity of XML systems and the demand for large-scale batch production of publication-quality typesetting from such sources has seen a steady increase in the use of LaTeX.


\begin{table}
\begin{center}

\begin{tabular}{|c|c|c|c|c|c|c|}
\hline 
&
\multicolumn{2}{c|}{Some Data}&
&
\multicolumn{2}{c|}{Some More Data}&
\tabularnewline
\hline
\hline 
&  Hi-Res&  Lo-Res&  Reduction&  Hi-Res&  Lo-Res&  Speedup
\tabularnewline
\hline 
Row Data A &  225,467&  43,850&  80.6\%&  360&  90&  4.0
\tabularnewline
\hline 
Row Data B &  225,467&  16,388&  92.7\%&  360&  26&  13.8
\tabularnewline
\hline 
\end{tabular}


\captionfonts
\caption[Performance data]{Here is some performance data for the system.}
\label{table:performance}
\end{center}
\end{table}


LaTeX is a document markup language and document preparation system for the TeX typesetting program.

It is widely used by mathematicians, scientists, philosophers, engineers, and scholars in academia and the commercial world, and by others as a primary or intermediate format (e.g. translating DocBook and other XML-based formats to PDF) because of the quality of typesetting achievable by TeX. It offers programmable desktop publishing features and extensive facilities for automating most aspects of typesetting and desktop publishing, including numbering and cross-referencing, tables and figures, page layout and bibliographies.

LaTeX is intended to provide a high-level language to access the power of TeX. LaTeX essentially comprises a collection of TeX macros, and a program to process LaTeX documents. Since TeX's formatting commands are very low-level, it is usually much simpler for end-users to use LaTeX.

LaTeX was originally written in 1984 by Leslie Lamport at SRI International and has become the dominant method for using TeX�few people write in plain TeX anymore.

LaTeX is based on the idea that authors should be able to focus on the meaning of what they are writing, without being distracted by the visual presentation of the information. In preparing a LaTeX document, the author specifies the logical structure using familiar concepts such as chapter, section, table, figure, etc., and lets the LaTeX system worry about the presentation of these structures. It therefore encourages the separation of layout from content, while still allowing manual typesetting adjustments where needed. This is similar to the mechanism by which many word processors allow styles to be defined globally for an entire document, or the CSS mechanism used by HTML.

LaTeX can be arbitrarily extended by using the underlying macro language to develop custom formats. Such macros are often collected into packages which are available to address special formatting issues such as complicated mathematical content or graphics. In addition, there are numerous commercial implementations of the entire TeX system, including LaTeX, to which vendors may add extra features like additional typefaces and telephone support. LyX is a free visual document processor that uses LaTeX for a back-end. TeXmacs is a free, WYSIWYG editor with similar functionalities as LaTeX, but a different typesetting engine.

A number of popular commercial desktop publishing systems use modified versions of the original TeX typesetting engine. The recent rise in popularity of XML systems and the demand for large-scale batch production of publication-quality typesetting from such sources has seen a steady increase in the use of LaTeX.


% ------------- End main chapters ----------------------

\clearpage
\bibliography{bibliography}
\bibliographystyle{plain}
%\addcontentsline{toc}{chapter}{Bibliography}

\end{document}